% English Article template created by Vopaaz
\documentclass{article}
\usepackage{geometry}
\geometry{a4paper}
\usepackage{setspace}
\usepackage{enumerate}
\usepackage{enumitem}
\usepackage{hyperref}
\hypersetup{colorlinks,allcolors=black}

\setenumerate[1]{itemsep=0pt,partopsep=2pt,parsep=0pt ,topsep=2pt}
\setitemize[1]{itemsep=0pt,partopsep=2pt,parsep=0pt ,topsep=2pt}
\setenumerate[2]{itemsep=0pt,partopsep=2pt,parsep=0pt ,topsep=2pt}
\setitemize[2]{itemsep=0pt,partopsep=2pt,parsep=0pt ,topsep=2pt}
\setdescription{itemsep=0pt,partopsep=2pt,parsep=0pt ,topsep=2pt}

\usepackage{graphicx}
\usepackage{fontspec}

\defaultfontfeatures{%
	RawFeature={%
		+swsh,
		+calt
	}%
}

\setmainfont{EB Garamond}

%-----------%

\title{Introduction to Artificial Intelligence Writing 5}
\author{YiFan Li\\ZeYuan Yang}
\date{\today}



\begin{document}

\addfontfeatures{RawFeature={+smcp}}
\maketitle
\addfontfeatures{RawFeature={-smcp}}

%-------%


\section{Introduction}

Predicting the future can only happen in science fictions,
but predicting future trend of some certain time series is feasible.
Time series analysis has a long history, and multiple approaches have been studied.

The most traditional methods are statistical ones such as the well-known ARIMA model.
Although the model is simple, it is still used in recent researches such as \cite{arima-bus-travel}.
Some delicately designed machine learning approaches are also introduced in these years,
varying from relatively simple support vector machine \cite{support-vector-machine}
to complex neural network \cite{time-series-prediction-and-neural-networks}.

The one that receives most attention among all time series is probably the stock price.
Firstly, as an essential part of financial market, stock market is the apple of investors' eyes.
Not only financial researchers but also individuals and securities companies have
tried their best to analyze the stock price time series and predict its behavior in the future,
in order to make a fortune.
Secondly, unlike most time series prediction tasks where the output must be exact future values,
only knowing whether the future price will go up or down is already sufficient to make a trading decision.
Therefore, the problem can be simplified thus enabling more ideas and approaches to be applicable.
Thirdly, sufficient financial indexes can serve as domain knowledge to be combined with computer algorithms.

Nowadays, over two hundred different financial rules were designed to
suggest buying, selling or holding decisions based on historical stock price time series \cite{stock-timing-using-genetic-algorithms}.
Each rule has its own rationality and will definitely indicate some features of the target stock.
However, the movement of stock price is the combined result of massive number of microeconomic and macroeconomic factors,
which makes it impossible to predict the trend with any index solely.
As Korczak \cite{stock-timing-using-genetic-algorithms} pointed out,
there is no rule consistently outperforming the others.
Therefore, finding the most appropriate way to combine them becomes the key to the problem.

Although some scholars argue that stock prices follow a random walk pattern,
most experts in this field believe that the prices can be predicted with more than 50 percent accuracy,
with eligible financial indexes and reasonable combination \cite{stock-market-prediction-with-multiple-classifiers}.
Plenty of approaches has been examined and we will go through some of them in the next section.
Noticeably, genetic algorithm is widely used in related researches.

% for instance, neutral networks used by Kimoto et al \cite{stock-market-prediction-system-with-modular-neural-networks}.
% Genetic algorithm is one of them as well.

In our paper, we will apply genetic algorithm to
find the best combination of a set of financial indexes, which can help with the buy and sell decisions.
We will dig deeper for a more comprehensive and applicable solution based on the previous works.

% Implementation details, could be useful in Section Method.

% ---------------- %
% We base on a historical stock price time series in the past \emph{n} periods
% to build an agent to make a final decision of buying and selling.
% % TODO: how many
% More than 10 different financial rules and historical prices information
% in the past three years of 5 stocks are included by us to build our experiment.

% Our experiment is designed to predict the stock price as a search problem.
% Every rule is assigned a weight.
% All these weights constitute a vector, a new, combination rule to predict the prices.
% Each vector is considered as a chromosome.
% Running genetic algorithm on this, we program to find the optimal prediction rule for stocks.
% The final selected agent is considered to have the capacity to make the most optimal buying and selling decisions.

% The experiment is conducted in two phases.
% We divide our dataset into two subsets, a train dataset and a test dataset.
% In the first stage, we will only use the train dataset to find the hyper parameters that lead to the best performance here.
% In the second stage, genetic algorithm will be run on the test dataset to measure performance.
% By this approach, we not only find the optimal solution, but also avoid the over-fitting problem.

% In the following section,
% we describe what other researchers have done in regard to the prediction in stock market,
% especially using genetic algorithm.

% ---------------- %


\section{Related Work}

The study of stock price prediction went through a long history.
Complicated algorithms and artificial intelligence were applied to the analysis at last century
\cite{application-of-neural-network-to-technical-analysis}.
But until nowadays, there is still not a sure card for this problem.
Neural networks, support vector regression and other algorithms domain this field
\cite{textual-analysis-of-stock-market-prediction}.
And genetic algorithm is one of them.

Due to the requirement to the data, some researchers choose to combine genetic and other approaches together.
Neural networks are one of them.
When developing a stock trading system, GFNN (genetic-algorithm-based fuzzy neural networks)
are applied to build the qualitative model.
But in these studies, genetic algorithm is only used to provide the initial weights
for the FNN (fuzzy neural networks) \cite{an-intelligent-stock-trading-decision-support-system}.
The core of the experiment is to learn and predict the stock market with the FNN.
This paper will not discuss much about this.
We will focus on studies with genetic algorithms as their key approach.

The basic background of the application of genetic algorithm is the stock price time series.
The experiment will vary on whether the time series are fuzzy or not.
If fuzzy time series, the percentage of price change needs to be converted into fuzzy value.
The fuzzy logic relationship and FLRs will be extracted and normalized to get a weight matrix
\cite{a-novel-stock-forecasting-model-based-on-fuzzy-time-series}.
Fitness functions can only be designed based on this weighted matrix.
The genetic algorithm will be run on this as well.

If the time series are not fuzzy, the study only needs to focus on the procedure of genetic algorithm.
Most experiments begin with choosing appropriate trading rules.
As mentioned, there're hundreds of existing trading rules,
which is impractical to run the genetic algorithm on all of them.
One approach for this is to categorize the rules and find representative ones.
Moving Average and Relative Strength Index are the most common rules.
Some other indexes are representative as well.
For example, \emph{rate of change} represents the price trend between current period and the past \emph{n} period.
However, only by experimenting and testing can we find which rules are helpful to making decisions.

The chromosomes are constructed based on the trading rules.
They consist of the weight of the selected rules.
For a single stock price time series, each trading rule will indicate a decision for buying or selling.
Assigning bits and weight on the rules and then combine them together, we can get the chromosome.
In Fuentu's study in 2006, three main rules are selected, and five indexes are coded in the algorithm
\cite{genetic-algorithms-to-optimise-the-time-to-make-stock-market-investment}.
The experiment assigned 3 bits for each index and construct a chromosome of 15 bits long.
While some other experiments use only one binary bit to represent one financial rule
\cite{stock-timing-using-genetic-algorithms}.
% TODO: what about ours

The mutation and crossover process are typical genetic algorithm since we got the chromosome.
Some variants of genetic algorithm have different attitude to mutation rate and crossover rate.
But we have not found study designing the experiment based on them.
As for the fitness function, each agent applies their combination rules to make decisions.
In Badawy's study, he uses \emph{1} to represent the buying decision, \emph{0.5} for doing nothing and \emph{0} for selling,
based on the rules of the agent and the period it analyzing
\cite{genetic-algorithms-for-predicting-the-egyptian-stock-market}.
With the decisions of each agent, we can evaluate the performance of them and choose the qualified chromosome.

Running with appropriate evaluation function and fitness function,
there're always some combinations constantly performing better.
These winners are what we believe to be the best combinations for making buying and selling decisions.
But there still two concerns about this approach.

Fist, experiment is always run on limited stocks.
Whether our result is suitable for the majority stocks in the market requires observation.
Even some experts argue that one set of rules found by algorithms are not efficient even in a group of stocks
\cite{stock-timing-using-genetic-algorithms}.
The second concern is whether the result is useful as time goes by.
Our experimental base is the historical price data.
If the historical data exceeds, the result may vary.
However, there are also some opinions that the influence of time is not obvious
\cite{stock-timing-using-genetic-algorithms}.

Our work designs the experiment similar as the mentioned approach,
digging deeper into the rules, adopting variants of genetic algorithms and running in two phases.
In doing so, we will be able to avoid some common problems and discover practical results.


\bibliographystyle{abbrv}
\bibliography{report}


%-------%



\end{document}



